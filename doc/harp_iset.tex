\documentclass[10pt,letterpaper]{article}

\usepackage[margin=1in]{geometry}
\usepackage{graphicx}
\title{HARP Instruction Set Manual}
\author{Chad D. Kersey}

\begin{document}
\maketitle
\section*{Disclaimer}
This document, like the work it documents, is very much a work in progress.
Send any corrections, updates, suggestions, or complaints (preferably in patch form, this file is under \texttt{harptool/doc} in the HarpTool repo) to \texttt{cdkersey@gatech.edu}.

\section*{Introduction}
HARP is two things, a multi-year Heterogeneous Architecture Research Project, and an implementation of a specific Heterogeneous Architecture Research Prototype.
It is for the latter that the HARP instruction set architectures have been created.
This is a space of SIMT(GPU) oriented RISC-like instruction sets with the following properties:

\begin{itemize}
  \item{Full predication}
  \item{Assembly language level compatibility}
  \item{SIM[DT] parallelism}
  \item{Little endianness}
  \item{8-bit byte size}
  \item{Customizability}
\end{itemize}

The customizability of the HARP ISAs is illustrated by facts missing from this list of features.
The data path width, instruction encodings, number of registers (general purpose and predicate) are all left up to the implementation.
Harptool, the HARP assembler, linker, emulator, and disassembler, is passed information about the ISA through an architecture identifier string, or \texttt{ArchID}.
An \texttt{ArchID} uniquely identifies a HARP ISA.

\section{Architecture Identifier String (\texttt{ArchID})}
The best way to understand the multifaceted parameterizablity of the HARP ISAs is to study the architecture identifier strings used to uniquely identify a single HARP instruction set architecture.
We'll start by breaking down Harptool's default \texttt{ArchID}: \texttt{8w32/32/8}:

\begin{center}
\begin{tabular}{rl}
\textbf{Field}&\textbf{Meaning}\\
\hline
\texttt{8} &8-byte (64-bit) registers and addresses\\
\texttt{w} &Word-based (64-bit) fixed-width instruction encoding\\
\texttt{32}&32 general-purpose registers per lane\\
\texttt{32}&32 predicate registers per lane\\
\texttt{8} &8 SIMD lanes\\
\end{tabular}
\end{center}

All ArchIDs have a similar format, although the SIMD lanes field can be omitted, as object files are still fully compatible even if the number of lanes changes. 

\section{HarpTool}
The assembler/linker/emulator/disassembler program for HARP is called HarpTool.
It is a multiple-function executable, its function selected with the first command line argument.
When run with no command line arugments the HarpTool executable prints a help message explaining the available command line arguments.

All of the HARP utilities can take an archID as a command line parameter.
If none is provided, a default will be assumed.

\subsection{Assembler}
The assembler converts assembly files to HOF, the Harp Object Format.

\subsection{Linker}
The linker combines HOF files and produces raw RAM images for use by the emulator.
An intended future use is the conversion of multiple HOF files to statically-linked HOF executables.

\subsection{Emulator}
The emulator emulates a HARP system.
It can be provided with a 

\subsection{Disassembler}
The disassembler is used to convert HOF files to equivalent assembly files.
One of its intended uses is the conversion of HOF object files between different HARP ISAs, say from 8w32/32 to 8b32/32.

\subsection{Test Programs}
In the \texttt{harptool/test} directory there is a set of test programs.
The makefile in this directory assembles, links, and emulates them, placing the output in plain text files.

\subsubsection{\texttt{hello.s}}
The simplest example prints a message and exits.

\subsubsection{\texttt{2thread.s}}
\texttt{2thread} performs a vector addition across two threads.

\subsubsection{\texttt{sieve.s}}
\texttt{sieve} performs the Sieve of Eratosthenes in a single thread and prints the results, including the count of total prime numbers found.

\section{Instruction Encoding}
There are two currently-supported types of instruction encoding, but they all share a similar basic structure.
The opcodes and types of fields required by each instruction are identical, differentiated only by the number of bits available for each type of field and the way predication is specified.

\subsection{Argument Classes}
Instructions can be broadly categorized by the types of arguments they require.
The bit fields in the instruction encodings depend heavily on this quality.

\begin{center}
\begin{tabular}{c|c|l}
\textbf{Argument Class}&\textbf{Description}&\textbf{Example}\\
\hline
\texttt{AC\_NONE}    &No arguments             &\texttt{di;}                 \\
\texttt{AC\_2REG}    &2 GPRs, 1 in, 1 out      &\texttt{neg    \%r1, \%r2;}  \\
\texttt{AC\_2IMM}    &1 immediate in, 1 GPR out&\texttt{ldi    \%r1, \#0xff;}\\
\texttt{AC\_3REG}    &3 GPRs, 2 in, 1 out  &\texttt{add    \%r1, \%r2, \%r2;}\\
\texttt{AC\_3PREG}   &3 pred. regs, 2 in, 1 out&\texttt{andp   @p0, @p0, @p1;}\\
\texttt{AC\_3IMM} &GPR in, imm. in, GPR out &\texttt{andi   \%r1, \%r3, \#3;}\\
\texttt{AC\_3REGSRC} &3 GPRs in            &\texttt{tlbadd \%r0, \%r1, \%r2;}\\
\texttt{AC\_1IMM}    &1 imm in             &\texttt{jmpi   label;}          \\
\texttt{AC\_1REG}    &1 reg in             &\texttt{jmpr   \%r2}            \\
\texttt{AC\_3IMMSRC} &2 GPRs in, 1 imm. in &\texttt{st     \%r1, \%r2, \#10;}\\
\texttt{AC\_PREG\_REG}&GPR in, pred. reg. out &\texttt{iszero @p0, \%r3;}    \\
\texttt{AC\_2PREG}   &2 pred. regs, 1 in, 1 out&\texttt{notp   @p0, @p0;}    \\
\end{tabular}
\end{center}

\subsection{Opcode/Instruction Class Table}
\begin{verbatim}
  00  "nop"      NONE      01  "di"       NONE      02  "ei"       NONE 
  03  "tlbadd"   3REGSRC   04  "tlbflush" NONE      05  "neg"      2REG     
  06  "not"      2REG      07  "and"      3REG      08  "or"       3REG     
  09  "xor"      3REG      0a  "add"      3REG      0b  "sub"      3REG     
  0c  "mul"      3REG      0d  "div"      3REG      0e  "mod"      3REG
  0f  "shl"      3REG      10  "shr"      3REG      11  "andi"     3IMM     
  12  "ori"      3IMM      13  "xori"     3IMM      14  "addi"     3IMM     
  15  "subi"     3IMM      16  "muli"     3IMM      17  "divi"     3IMM     
  18  "modi"     3IMM      19  "shli"     3IMM      1a  "shri"     3IMM     
  1b  "jali"     2IMM      1c  "jalr"     2REG      1d  "jmpi"     1IMM
  1e  "jmpr"     1REG      1f  "clone"    1REG      20  "jalis"    3IMM
  21  "jalrs"    3REG      22  "jmprt"    1REG      23  "ld"       3IMM
  24  "st"       3IMMSRC   25  "ldi"      2IMM      26  "rtop"     PREG_REG
  27  "andp"     3PREG     28  "orp"      3PREG     29  "xorp"     3PREG    
  2a  "notp"     2PREG     2b  "isneg"    PREG_REG  2c  "iszero"   PREG_REG 
  2d  "halt"     NONE      2e  "trap"     NONE      2f  "jmpru"    1REG
  30  "skep"     1REG      31  "reti"     NONE      32  "tlbrm"    1REG
  33  "itof"     2REG      34  "ftoi"     2REG      35  "fadd"     3REG
  36  "fsub"     3REG      37  "fmul"     3REG      38  "fdiv"     3REG
  39  "fneg"     2REG
\end{verbatim}

\subsection{Word Encoding}

Word-based instruction encodings all share the initial fields:
\begin{itemize}
  \item The most-significant bit is 1 if the instruction is predicated and 0 otherwise.
  \item The next $log_2(\mathrm{\#pred\_regs})$ specify the predicate register.
  \item The next 6 bits are used for the opcode.
\end{itemize}

After this, the operands of the instruction are ordered corresponding to their ordering in the assembly language, sized according to the following rules:
\begin{itemize}
  \item Register operands are $log_2(\mathrm{\#GPRs})$ bits long, or just enough bits to uniquely identify a register.
  \item Predicate register operands are $log_2(\mathrm{\#pred\_regs})$ bits long, or just enough bits to uniquely identify a predicate register.
  \item Immediate fields are always the last field and occupy the remaining bits of the instruction. All immediate fields are sign extended to the length of a machine word.
\end{itemize}

\begin{center}
\includegraphics{fig/word_enc}
\end{center}

\subsection{Byte Encoding}
In the byte encoding, each field of the instruction (predicate, opcode, operands) occupies a byte, with the exception of immediates, which occupy an unaligned word.
All instructions have a predicate and opcode byte.
The predicate byte is all ones if the instruction is not predicated; oterwise the predicate byte contains the predicate register number used to predicate the instruction.
Just like the word-based instruction encoding, registers appear in the same order as the assembly language, destination-first.

\begin{center}
\includegraphics{fig/byte_enc}
\end{center}

\section{Assembly Language}
The assembly language is fairly easy to pick up from the HarpTools examples. It is RISC-like, and written destination register first (in this it differs from Unix assembly syntax).
Registers names are prefixed with the percent sign (\%) and predicate register names with the at symbol (@).
Predicated instructions are prefixed with the predicate register name and a question mark:
\begin{verbatim}
  @p0 ? addi %r7, %r1, #1
\end{verbatim}
A small set of directives is provided to express non-instruction data:

\begin{center}
\begin{tabular}{cl}
\textbf{Directive}&\textbf{Use}\\
\hline
\texttt{.align 256}    &Align next symbol to a multiple of 256 bytes.\\
\texttt{.word 0x1234}  &Insert a word with the value \texttt{0x1234}.\\
\texttt{.byte 0xff}    &Insert a byte with the value \texttt{0xff}.\\
\texttt{.def SYM 123}  &Replace SYM with 123 in immediate operands.\\
\texttt{.entry}        &Make the next label the HOF executable entry point.\\
\texttt{.global}       &Give the next label global (external) linkage.\\
\texttt{.perm rw}      &Set HOF permissions of the next label to read/write.\\
\texttt{.string "Str"} &Create a null terminated string.\\
\end{tabular}
\end{center}

\section{Instruction Set}
\subsection{Trivial Instruction}
\begin{center}
\begin{tabular}{cl}
\textbf{Instruction}&\textbf{Description}\\
\hline
\texttt{nop}&No operation.\\
\end{tabular}
\end{center}

\subsection{Privileged Instructions}
\begin{center}
\begin{tabular}{cl}
\textbf{Instruction}&\textbf{Description}\\
\hline
\texttt{ei}      &Enable interrupts.\\
\texttt{di}      &Disable interrupts.\\
\texttt{skep} \%addr&Set kernel entry point.\\
\texttt{tlbadd} \%virt, \%phys, \%flags&Add an entry to the TLB.\\
\texttt{tlbrm} \%virt&Remove entry corresponding to virt. address from TLB.\\
\texttt{tlbflush}&Remove all but default entry from TLB.\\
\texttt{jmpru} \%addr&Jump indirect and switch to user mode.\\
\texttt{reti}&Return from interrupt.\\
\texttt{halt}&Halt CPU until next interrupt.\\
\end{tabular}
\end{center}

The flags register used by \texttt{tlbadd} stores, in its least-significant four bits, in order from most to least significant:
\begin{center}
\begin{tabular}{cl}\
\textbf{Bit}&\textbf{Meaning}\\
\hline
  \texttt{kx}&Kernel can execute.\\
  \texttt{kw}&Kernel can write.\\
  \texttt{kr}&Kernel can read.\\
  \texttt{ux}&User can execute.\\
  \texttt{uw}&User can write.\\
  \texttt{ur}&User can read.\\
\end{tabular}
\end{center}

\subsection{Memory Loads/Stores}
\begin{center}
\begin{tabular}{cl}
\textbf{Instruction}&\textbf{Description}\\
\hline
\texttt{st} \%val, \%base, \textsc{\#Offset}&Store.\\
\texttt{ld} \%dest, \%base, \textsc{\#Offset}&Load.\\
\end{tabular}
\end{center}

\subsection{Predicate Manipulation}

\begin{center}
\begin{tabular}{cl}
\textbf{Instruction}&\textbf{Description}\\
\hline
\texttt{andp} @dest, @src1, @src2&Logical and.\\
\texttt{orp} @dest, @src1, @src2&Logical or.\\
\texttt{xorp} @dest, @src1, @src2&Exclusive or.\\
\texttt{notp} @dest, @src&Complement.\\
\end{tabular}
\end{center}

\subsection{Value Tests}

\begin{center}
\begin{tabular}{cl}
\textbf{Instruction}&\textbf{Description}\\
\hline
\texttt{rtop} @dest, \%src&Set @dest if \%src is nonzero.\\
\texttt{isneg} @dest, \%src&Set @dest if \%src is negative.\\
\texttt{iszero} @dest, \%src&Set @dest if \%src is zero.\\
\end{tabular}
\end{center}

\subsection{Immediate Integer Arithmetic/Logic}

\begin{center}
\begin{tabular}{cl}
\textbf{Instruction}&\textbf{Description}\\
\hline
\texttt{ldi} \%dest, \textsc{\#Imm}&Load immediate.\\
\texttt{addi} \%dest, \%src1, \textsc{\#Imm}&Add immediate.\\
\texttt{subi} \%dest, \%src1, \textsc{\#Imm}&Subtract immediate.\\
\texttt{muli} \%dest, \%src1, \textsc{\#Imm}&Multiply immediate.\\
\texttt{divi} \%dest, \%src1, \textsc{\#Imm}&Divide immediate.\\
\texttt{modi} \%dest, \%src1, \textsc{\#Imm}&Modulus immediate.\\
\texttt{shli} \%dest, \%src1, \textsc{\#Imm}&Shift left immediate.\\
\texttt{shri} \%dest, \%src1, \textsc{\#Imm}&Shift right immediate.\\
\texttt{andi} \%dest, \%src1, \textsc{\#Imm}&And immediate.\\
\texttt{ori} \%dest, \%src1, \textsc{\#Imm}&Or immediate.\\
\texttt{xori} \%dest, \%src1, \textsc{\#Imm}&Xor immediate.\\
\end{tabular}
\end{center}

\subsection{Register Integer Arithmetic/Logic}
\begin{center}
\begin{tabular}{cl}
\textbf{Instruction}&\textbf{Description}\\
\hline
\texttt{add} \%dest, \%src1, \%src2&Add.\\
\texttt{sub} \%dest, \%src1, \%src2&Subtract.\\
\texttt{mul} \%dest, \%src1, \%src2&Multiply.\\
\texttt{div} \%dest, \%src1, \%src2&Divide.\\
\texttt{mod} \%dest, \%src1, \%src2&Modulus.\\
\texttt{shl} \%dest, \%src1, \%src2&Shift left.\\
\texttt{shr} \%dest, \%src1, \%src2&Shift right.\\
\texttt{and} \%dest, \%src1, \%src2&And.\\
\texttt{or} \%dest, \%src1, \%src2&Or.\\
\texttt{xor} \%dest, \%src1, \%src2&Xor.\\
\texttt{neg} \%dest, \%src1, \%src2&Two's complement.\\
\texttt{not} \%dest, \%src1, \%src2&Bitwise complement.\\
\end{tabular}
\end{center}

\subsection{Floating Point Arithmetic}
These operations operate on real numbers in an implementation-determined
format, which can be fixed point or floating point.

\begin{center}
\begin{tabular}{cl}
\textbf{Instruction}&\textbf{Description}\\
\hline
\texttt{itof} \%dest, \%src&Integer to floating point.\\
\texttt{ftoi} \%dest, \%src&Floating point to integer.\\
\texttt{fneg} \%dest, \%src&Negate (complement sign bit).\\
\texttt{fadd} \%dest, \%src1, \%src2&Floating point add.\\
\texttt{fsub} \%dest, \%src1, \%src2&Floating point subtract.\\
\texttt{fmul} \%dest, \%src1, \%src2&Floating point multiply.\\
\texttt{fdiv} \%dest, \%src1, \%src2&Floating point divide.\\
\end{tabular}
\end{center}

\subsection{Control Flow}

\begin{center}
\begin{tabular}{cl}
\textbf{Instruction}&\textbf{Description}\\
\hline
\texttt{jmpi} \textsc{\#RelDest}&Jump to immediate (PC-relative).\\
\texttt{jmpr} \%addr&Jump indirect.\\
\texttt{jali} \%link, \textsc{\#RelDest}&Jump and link immediate.\\
\texttt{jalr} \%link, \%reg&Jump and link indirect.\\
\end{tabular}
\end{center}

\subsection{SIMD Control}

\begin{center}
\begin{tabular}{cl}
\textbf{Instruction}&\textbf{Description}\\
\hline
\texttt{clone} \%lane&Clone register state into specified lane.\\
\texttt{jalis} \%link, \%n, \textsc{\#RelDest}&Jump and link immediate, spawning N active lanes.\\
\texttt{jalrs} \%link, \%n, \%dest&Jump and link indirect, spawning N active lanes.\\
\texttt{jmprt} \%addr&Jump indirect, terminating execution on all but a single lane.\\
\end{tabular}
\end{center}

\subsection{User/Kernel Interteraction}

\begin{center}
\begin{tabular}{cl}
\textbf{Instruction}&\textbf{Description}\\
\hline
\texttt{trap}&User-generated interrupt.\\
\end{tabular}
\end{center}

\section{Interrupts}
The HARP interrupt mechanism is simple.
For SIMD lane 0, there is a shadow register file, program counter, and active lane count.
When an interrupt occurs, the state of lane zero is saved into these shadow registers, and execution resumes at the kernel entry point.
The type of interrupt is specified by the value placed in register 0 at this time, according to the following table:

\begin{center}
\begin{tabular}{cl}
\textbf{Number}&\textbf{Description}\\
\hline
0    &Trap (user-generated interrupt)\\
1    &Page fault due to absence from TLB\\
2    &Page fault due to permission violation\\
3    &Invalid/unsupported instruction\\
4    &Divergent branch\\
5    &Numerical domain (divide by zero)\\
6-7&(reserved for future exceptions)\\
8    &Console input\\
\end{tabular}
\end{center}

The first eight interrupt numbers are reserved for internal CPU-generated exceptions, and all of the remaining numbers are free for use by hardware.

\section{Application Binary Interface}
The ABI assumes a set of at least four general purpose registers.
The frame pointer is optional and can be stored on the stack itself if necessary.
The stack pointer and link register, in this order, are always the two highest-numbered registers.
If 8 or more registers are available, the frame pointer may be the register one less than the register number of the stack pointer.

\begin{itemize}
  \item The lower-numbered half of the registers are caller-saved (temporary).
  \item The upper-numbered half are therefore callee-saved.
  \item The callee is responsible for adjusting the stack and frame pointers, if such adjustment is required.
  \item The stack grows toward smaller addresses (subtract to push, add to pop).
  \item Pointer function arguments and numerical arguments that can fit in a single register are passed through temporary registers, starting with register \texttt{\%r0}. If more registers are required than there are temporary registers available, stack space at addresses less than the stack pointer is used.
  \item Record (struct) return values and numerical return values larger than the word size are always passed on the stack. The caller is responsible for allocating the necessary space. The stack pointer at the time of call is a pointer to the returned structure. All other return values are returned in \texttt{\%r0}.
\end{itemize}

\section{SIMD Operation}
The HARP ISAs are inherently SIMD.
In addition to designs with a single set of functional units and architectural registers, designs are allowed that replicate these while retaining a single front-end and memory system.
This allows for multiple threads executing the same stream of instructions to simultaneously occupy multiple ``lanes'' of the processor.
When a predicated control flow instruction occurs without unanimous agreement among predicate registers, a divergent branch has occurred.
The current response to this is to trap to the operating system (interrupt number 4).

%\begin{center}
%\includegraphics{fig/simd_core}
%\end{center}

\subsection{Instructions for SIMD Operation}
The \texttt{clone}, \texttt{jalis}, \texttt{jalrs}, and \texttt{jmprt} instructions form the basis of SIMD context control in the HARP instruction set.
Context is created using \texttt{clone}, the waiting threads are spawned using \texttt{jalrs} or \texttt{jalis}, ``jump-and-link immediate/register and spawn'', and finally the parallel section returns using \texttt{jmprt}, ``jump register and terminate'', best thought of as ``return and terminate.''

\section{Default I/O Devices}
The emulator currently only supports a single I/O device, simple console I/O.
Writing to the address \texttt{0x800...0} (an address with its MSB set and all other bits cleared) causes text to be written to the display.
Input on this console interface causes an interrupt (number 8).

\end{document}
